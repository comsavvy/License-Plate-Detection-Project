\documentclass[aspectratio=169]{beamer}

\usetheme{Madrid}
\usecolortheme{default}
\usefonttheme{professionalfonts}
\setbeamertemplate{navigation symbols}{}

\title[Plate Detection]{Senegalese License Plate Detection System}
\subtitle{Manual Parsing and Regex Approaches}
\author{Group 11: Alliance, Maryam, Olusola, Jean}
\institute{AIMS Rwanda}
\date{\today}

% Helper commands
\newcommand{\code}[1]{\texttt{#1}}

\begin{document}

%====================================================
\begin{frame}
  \titlepage
\end{frame}

%====================================================
\begin{frame}{Agenda}
  \tableofcontents[hidesubsections]
\end{frame}

%====================================================
\section{Problem Overview}

\begin{frame}{Motivation}
  \begin{itemize}
    \item Extract structured information (license plates) from plain text.
    \item Useful for logs, reports, monitoring, and preprocessing.
    \item Focus: Senegalese vehicle plate formats only.
    \item Two complementary methods: manual parser and regex pattern.
  \end{itemize}
\end{frame}

\begin{frame}{Plate Formats}
  Accepted canonical patterns (letters A--Z, digits 0--9):
  \begin{itemize}
    \item \code{XY-1234-T}
    \item \code{XY-1234-ZT}
  \end{itemize}
  Separators may be hyphen or single space in input: \code{XY 1234 T}.
\end{frame}

\begin{frame}{Rules Recap}
  \begin{itemize}
    \item 2 starting letters
    \item Separator: \code{-} or space
    \item 4 digits
    \item Separator: \code{-} or space
    \item 1 or 2 trailing letters
    \item Not embedded inside longer alphanumeric strings
    \item Case-insensitive input; uppercase normalized output
  \end{itemize}
\end{frame}

%====================================================
\section{Objectives}

\begin{frame}{Core Objectives}
  \begin{itemize}
    \item Detect all valid plates in arbitrary text.
    \item Print normalized unique results in order.
    \item Report when none are found.
    \item Provide simple, readable logic.
  \end{itemize}
\end{frame}

\begin{frame}{Extended Objectives}
  \begin{itemize}
    \item Allow punctuation around plates.
    \item Interactive menu (manual testing).
    \item Provide alternative regex method.
    \item Keep code dependency-free (core version).
  \end{itemize}
\end{frame}

%====================================================
\section{Manual Parsing Approach}

\begin{frame}{Why Manual Parsing?}
  \begin{itemize}
    \item Educational transparency.
    \item Fine control over each rule.
    \item Easy to extend for new patterns.
    \item Avoids reliance on regex engine.
  \end{itemize}
\end{frame}

\begin{frame}[fragile]{Pseudocode (High-Level)}
\begin{verbatim}
UPPER = text uppercased
plates = empty list
i = 0
while i < length - minimal_length:
  if next 2 chars not letters: i++ ; continue
  if next sep not '-' or ' ': i++ ; continue
  if next 4 chars not digits: i++ ; continue
  if next sep not '-' or ' ': i++ ; continue
  read 1 or 2 letters as end_part
  if none: i++ ; continue
  check boundary before and after
  if ok:
     plate = canonical form with hyphens
     store if new
     i = end of match
  else:
     i++
if plates empty: report none else print list
\end{verbatim}
\end{frame}

\begin{frame}{Boundary Handling}
  \begin{itemize}
    \item Ensure no letter/digit directly touches start or end of detected pattern.
    \item Prevents false matches inside longer tokens (e.g. \code{AAXY-1234-T}).
    \item Accepts punctuation (comma, period) near plates.
  \end{itemize}
\end{frame}


%====================================================
\section{Regex Approach}

\begin{frame}[fragile]{Regex Pattern}
  Core pattern (case-insensitive):
  \begin{block}{Pattern}
  \small\verb|(?i)(?<![A-Z0-9])([A-Z]{2})[- ](\d{4})[- ]([A-Z]{1,2})(?![A-Z0-9])|
  \end{block}
  \begin{columns}[T,totalwidth=\textwidth]
      \begin{column}{0.65\textwidth}
        \begin{block}{Pattern Components}
          \small
          \begin{itemize}
            \item \texttt{(?i)} Case-insensitive matching
            \item \texttt{(?<![A-Z0-9])} Left boundary (no letter/digit before)
            \item \texttt{([A-Z]{2})} Two letters
            \item \texttt{[- ]} Separator (hyphen or space)
            \item \texttt{(\d{4})} Four digits
            \item \texttt{([A-Z]{1,2})} One or two letters
            \item \texttt{(?![A-Z0-9])} Right boundary (no letter/digit after)
          \end{itemize}
        \end{block}
    \end{column}
    \begin{column}{0.35\textwidth}
      \begin{itemize}
        \vspace{2em}

        \item Lookbehind / lookahead enforce clean boundaries.
        \item Captures letter block, digits, ending letters.
        \item Short, expressive, fast.
      \end{itemize}
    \end{column}
  \end{columns}
\end{frame}

\begin{frame}{Manual vs Regex}
  \begin{tabular}{p{0.42\linewidth} p{0.52\linewidth}}
    \textbf{Manual} & \textbf{Regex} \\
    Stepwise logic & Compact expression \\
    Easy to tweak mid-steps & Faster to write \\
    Verbose & Dense syntax \\
    Didactic & Concise \\
  \end{tabular}
\end{frame}

%====================================================
\section{Testing and Results}

\begin{frame}{Test Design}
  Mixed sample paragraph included:
  \begin{itemize}
    \item Valid plates with hyphens and spaces.
    \item Single-letter and two-letter endings.
    \item Mixed casing.
    \item Near-miss invalid patterns.
  \end{itemize}
\end{frame}

\begin{frame}{Observed Outcomes}
  \begin{itemize}
    \item All expected valid plates detected.
    \item No false positives in sample.
    \item Output normalized consistently.
    \item Order preserved; duplicates removed.
  \end{itemize}
\end{frame}

%====================================================
\section{Limitations and Improvements}

\begin{frame}{Current Limitations}
  \begin{itemize}
    \item No fuzzy/typo tolerance.
    \item Only Senegal formats covered.
    \item Pattern validity only (not real issuance).
  \end{itemize}
\end{frame}

\begin{frame}{Potential Enhancements}
  \begin{itemize}
    \item Add fuzzy (edit-distance) matching.
    \item Multi-country format registry.
    \item Batch file / dataset scanning.
    \item JSON / CSV export.
    \item Performance benchmarking.
  \end{itemize}
\end{frame}

%====================================================
\section{Conclusion}

\begin{frame}{Conclusion}
  \begin{itemize}
    \item Two clear methods implemented: manual parser and regex.
    \item Both meet detection, normalization, and uniqueness goals.
    \item Manual path aids learning; regex path aids brevity.
    \item Solid base for future extensions (fuzzy logic, more formats).
  \end{itemize}
\end{frame}

\begin{frame}{Key Takeaways}
  \begin{itemize}
    \item Keep patterns explicit when teaching.
    \item Normalize early for consistency.
    \item Enforce boundaries to avoid false positives.
    \item Provide alternative implementations for flexibility.
  \end{itemize}
\end{frame}

%====================================================
\section*{Q&A}

\begin{frame}{Questions}
  \centering \Huge Thank You!\\[1em]
  \large Questions / Feedback?
\end{frame}

\end{document}